\documentclass{article}

% Basic packages - compatible with most systems
\usepackage[utf8]{inputenc}
\usepackage[T1]{fontenc}
\usepackage{hyperref}
\usepackage{graphicx}
\usepackage{amsmath}
\usepackage{amsfonts}
\usepackage{nicefrac}
\usepackage{geometry}
\usepackage{lineno}
\geometry{margin=1in, top=1in}

% Custom title formatting to match NeurIPS exactly
\makeatletter
\renewcommand{\maketitle}{%
  \begin{center}
    % Top thick line (4 points)
    \rule{\textwidth}{4pt}
    \vspace{1em}
    
    % Title (17 point, bold, centered)
    {\fontsize{17}{20}\selectfont\bfseries \@title \par}
    
    \vspace{1em}
    % Bottom thin line (1 point)
    \rule{\textwidth}{1pt}
    
    \vspace{2em}
    
    % Authors - simple format like NeurIPS
    \@author
    
    \vspace{1em}
  \end{center}
}
\makeatother

\title{Prosomoíosi (Simulation), an AI Documentary Film}

\author{
\textbf{Shuai Liu}$^1$, \textbf{Mar Canet Sola}$^{1,2}$\\
$^1$Academy of Media Arts Cologne (KHM), Cologne, Germany\\
$^2$BFM, Tallinn University, Estonia\\
\texttt{shuai.liu@khm.de, mar.canet@gmail.com}
}

\begin{document}

\maketitle

\begin{abstract}
This paper presents Prosomoíosi (Simulation), an AI documentary film that explores the philosophical and aesthetic implications of artificial intelligence in creative practice. Through a real-time generative AI pipeline, the work investigates the concept of "medium alignment" from a media-archaeological perspective, tracing technological evolution "from clay to code" to examine how successive simulation technologies overwrite cultural memory and identity. The work employs a détournement strategy, positioning AI both as a generative tool and as an object of critique, thereby exposing the dialectical relationship between authentic creation and pseudo-creation. By making algorithmic decision-making processes visible rather than hidden, we transform AI from a black-box generator into a transparent collaborator in the creative process. This approach addresses fundamental human anxieties about memory, identity, and cultural transmission in the AI era, proposing a symbiotic rather than competitive relationship between human creativity and machine intelligence. The work is presented in two formats: a 4K DCP version for theatrical screenings and a gallery edition for exhibition spaces, both maintaining consistent visual and auditory coherence.
\end{abstract}

\textbf{TL;DR:} We present an AI documentary film using transparent algorithms to explore human-AI creative collaboration, addressing how AI reshapes cultural memory while proposing symbiotic rather than competitive relationships.

\section{Description of the Work and the Roles of AI and ML}

\subsection{Overview of \textit{Prosomoíosi (Simulation)}}

\textit{Prosomoíosi (Simulation)} is an audiovisual work created through a real-time generative AI pipeline, advancing the concept of "medium alignment" through a media-archaeological perspective to explore how successive technological regimes continuously overwrite memory and identity. The work employs a détournement strategy—a contemporary media art form that remixes existing works to subvert their original meanings, selecting publicly familiar materials to achieve rapid and impactful transmission of contrasting information.

The work traces a media-archaeological narrative "from clay to code"—from ancient sand tables and armillary spheres to modern deep learning and generative adversarial networks. Each technological transformation represents not merely technical progress but a fundamental transformation in the relationship between humans and reality. Modeling technologies have evolved from cognitive tools to arbiters of truth, with large-scale AI networks now capable of autonomously constructing worlds from latent space and even guiding human decision-making.

The work is presented in two formats: (1) a 4K DCP version (24/50fps, REC.709, 5.1 surround sound) designed for theatrical or festival screenings; (2) a gallery edition using the same master in a loop, projected in 4K with four-channel audio. Both formats maintain consistent visual presentation, ensuring coherence across different exhibition environments.

\subsection{The Dual Role of AI and Machine Learning in Creation}

The relationship between artificial intelligence and aesthetics appears critically important in contemporary cultural contexts. Traditionally, aesthetics has been fundamentally regarded as a uniquely human domain, deriving meaning from subjective feelings and emotional responses closely tied to human culture, history, and personal perspectives. However, advances in artificial intelligence and its increasing participation in creative processes compel us to reconsider the essence of creativity concepts, preferences, and artistic communication.

Since the early 21st century, computational data analysis, machine learning, neural networks, and artificial intelligence have become increasingly integrated into the aesthetic domain. Artificial intelligence, as a widely used term with evolving definitions, utilizes advanced technologies to predict personal preferences, provide recommendations, and execute automatic editing. AI is now widely used to generate multimodal media convergence, with the aesthetic domain positioned as the ultimate assessment of AI capabilities and constraints.

In \textit{Prosomoíosi}, AI and machine learning assume a dual role: serving both as generative tools and as objects of critique, embodying our reflection on the dialectical relationship between "authentic creation and pseudo-creation."

\subsection{Philosophical Analysis of Authentic Creation versus Pseudo-Creation}

Through analysis of human aesthetics and the broader aesthetic culture of human existence, we believe AI systems have the capacity to generate new cultures built upon human culture but producing entirely new aesthetic systems through artificial perspectives~\cite{brinkmann2023machine}. This phenomenon, termed "machine culture" by recent research, describes culture that is mediated or generated by machines, representing not only a transformation of human culture but also its evolution into new forms where intelligent machines participate in cultural variation, transmission, and selection~\cite{brinkmann2023machine}. Nevertheless, whether artificial intelligence truly creates these new cultures or merely demonstrates a form of pseudo-creation remains a contentious topic.

This debate can be traced back to Plato's philosophical thinking. In Plato's works, the term phantasm relates to deception or imagination and is often used critically. Plato primarily explored this viewpoint in \textit{The Republic}, distinguishing between two types of image-making: visual arts that produce accurate depictions of reality (similar to photography), and phantasm that produces false or misleading images (similar to AI-generated images).

AI-generated art often evokes uncanny feelings due to its homogenized aesthetic—lacking uniqueness and displaying repetitive stylistic elements. This "pseudo-creation" phenomenon stems from algorithmic patterns that audiences instinctively recognize as artificial. Our work exposes these mechanisms, making AI's decision-making processes visible rather than hidden, transforming AI from mere tool into medium for cultural critique.By exposing algorithmic decision-making processes, we invite audiences to become witnesses and participants in this creative process, thereby renegotiating the relationship between human creativity and machine intelligence. This transparency transforms AI from a black-box generator into a visible negotiator of meaning.

\section{Description on How the Theme of Humanity is Addressed}

\subsection{AI as Creative Partner Rather Than Replacement}

The potential of artificial intelligence as a creative tool lies in assisting and inspiring rather than replacing uniquely human creativity. As the core of aesthetics lies in personal subjective experience, the application of artificial intelligence should respect and enhance this uniqueness rather than eliminate its diversity and distinctive value.

\textit{Prosomoíosi} explores precisely this possibility of human-machine collaboration. We position AI as a visible collaborator rather than a hidden tool. By exposing algorithmic processes, we invite audiences to become witnesses and participants in this creative process, thereby renegotiating the relationship between human creativity and machine intelligence.

This approach embodies our fundamental concerns about humanity: In an era of rapid technological development, how do we maintain uniquely human aesthetic judgment and emotional experience? How do we ensure that AI becomes a tool that enhances rather than diminishes human creativity?

\subsection{Anxieties About Memory, Identity, and Cultural Transmission}

The core human themes confronted by the work are: How do successive simulation technologies overwrite cultural memory? What happens to human agency when algorithms mediate our understanding of history and identity? These are fundamental anxieties faced by contemporary humanity.

Simulation here is not simple copying or representation but power-laden rewriting produced through the interaction of technology, archives, and algorithms. Wolfgang Ernst notes that archival temporality shapes memory while quietly editing the past; Walter Benjamin foresaw the erosion of aura under mechanical reproduction; Jean Baudrillard later warned of simulacra supplanting reality. Our work precisely explores the new manifestations of these classical theories in the AI era.

\subsection{Renegotiating Ethical Relationships of Human-Machine Symbiosis}

We position humans and AI in symbiosis rather than competition. Following Bernard Stiegler's concept of technology as "third memory" and Yuk Hui's "cosmic technics," we view AI as an extension rather than replacement of human collective memory. The visible algorithmic processes in our work invite audiences to witness and participate in this extended cognition.

The core of this symbiotic relationship is redefining humanity itself: In the process of collaborating with AI, human unique value lies not in computational ability or information processing speed, but in emotional depth, cultural sensitivity, ethical judgment, and creative insight. Our work attempts to establish a new human-machine ethical relationship where technology enhances rather than threatens the core qualities of humanity.

\section{Author Biographies}

\textbf{Shuai Liu} is a Chinese digital media artist based in Cologne and Beijing. He earned a BFA in Digital Media Art from Guangzhou Academy of Fine Arts and is currently pursuing an MFA at the Academy of Media Arts Cologne. Since 2024, he has served as a research fellow at the Digital Humanities Research Center of Renmin University of China and as an advisor to Jinan University's Centre for AI and New Media Art.

Artist website: \url{https://liushuai.art}

\textbf{Mar Canet Sola} is a substitute professor at the Academy of Media Arts Cologne, a research fellow, and PhD candidate at Tallinn University’s Baltic Film, Media, and Arts School. His doctoral research, \textit{Exploring Artistic Strategies in Media Arts Using Generative AI}, investigates how artists employ deep learning within media art. He is a visiting researcher at the Center for Humans \& Machines at Max Planck. His artistic practice investigates generative systems, artificial intelligence, interactive art, and data-driven installations.

Artist website: \url{http://var-mar.info}



\begin{thebibliography}{10}
\bibitem{baudrillard} Baudrillard, J. (1994). \textit{Simulacra and Simulation}. University of Michigan Press.

\bibitem{benjamin} Benjamin, W. (1935). The Work of Art in the Age of Mechanical Reproduction. In \textit{Illuminations}, trans. Harry Zohn. Schocken Books.

\bibitem{bolter} Bolter, J. D., \& Grusin, R. (1999). \textit{Remediation: Understanding New Media}. MIT Press.

\bibitem{ernst} Ernst, W. (2013). \textit{Digital Memory and the Archive}. University of Minnesota Press.

\bibitem{han} Han, B.-C. (2015). \textit{The Burnout Society}, trans. Erik Butler. Stanford University Press.

\bibitem{harman} Harman, G. (2018). \textit{Object-Oriented Ontology: A New Theory of Everything}. Pelican Books.

\bibitem{hui} Hui, Y. (2016). \textit{On the Existence of Digital Objects}. University of Minnesota Press.

\bibitem{stiegler} Stiegler, B. (1998). \textit{Technics and Time, 1: The Fault of Epimetheus}. Stanford University Press.

\bibitem{manovich} Manovich, L. (2001). \textit{The Language of New Media}. MIT Press.

\bibitem{plato} Plato. \textit{The Republic}, trans. Benjamin Jowett. Oxford University Press.

\bibitem{brinkmann2023machine} Brinkmann, L., Baumann, F., Bonnefon, J. F., Derex, M., Müller, T. F., Nussberger, A. M., Czaplicka, A., Acerbi, A., Griffiths, T. L., Henrich, J., Leibo, J. Z., McElreath, R., Oudeyer, P. Y., Stray, J., \& Rahwan, I. (2023). Machine culture. \textit{Nature Human Behaviour}, 7(11), 1855-1868. \url{https://www.nature.com/articles/s41562-023-01742-2}
\end{thebibliography}

\end{document} 