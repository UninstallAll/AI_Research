\documentclass{article}

% Basic packages - compatible with most systems
\usepackage[utf8]{inputenc}
\usepackage[T1]{fontenc}
\usepackage{hyperref}
\usepackage{url}
\usepackage{amsfonts}

% Page layout to approximate NeurIPS style
\usepackage[letterpaper,margin=1in]{geometry}

% Simple title and section formatting
\usepackage{titlesec}
\titleformat{\section}{\large\bfseries}{\thesection}{1em}{}
\titleformat{\subsection}{\normalsize\bfseries}{\thesubsection}{1em}{}

\title{Artism: A Literature Review on Critical Research of Art Production in the AI Era}

\author{
\begin{tabular}{c c}
\begin{minipage}{0.45\textwidth}
\centering
Shuai Liu$^*$ \\
Academy of Media Arts Cologne (KHM) \\
Cologne, Germany \\
\texttt{shuai.liu@khm.de}
\end{minipage}
&
\begin{minipage}{0.45\textwidth}
\centering
Yiqing Tian$^*$ \\
Goldsmiths, University of London \\
London, UK \\
\texttt{yukitian321@gmail.com}
\end{minipage}
\end{tabular}\\
\vspace{0.3cm}
\begin{minipage}{\textwidth}
\centering
Mar Canet Sola$^*$ \\
BFM, Tallinn University, Estonia \\
Academy of Media Arts Cologne (KHM), Germany \\
\texttt{mar.canet@gmail.com} \\
\texttt{ORCID: 0000-0001-5986-3239}
\end{minipage}\\
\vspace{0.2cm}
$^*$ Equal contribution
}

\begin{document}

\maketitle

\begin{abstract}
This literature review provides theoretical foundations for the "Artism" research project, systematically examining contemporary art's conceptual dilemmas through the lens of artificial intelligence. The review investigates the "conceptual collage syndrome" in contemporary artistic production, analyzes the reconstruction of artistic aura in the AI era, and examines post-digital contexts and technological infrastructure. Through the case studies of AIDA (Artificial Intelligence Artist Database) and IsmismMachine, this research explores how AI-driven systems can both simulate and critically analyze artistic production, offering new perspectives on the relationship between technology and creativity in contemporary art practice.

\textbf{Keywords:} Artificial Intelligence Art, Conceptual Collage, Art Production, Multi-Agent Systems, Digital Humanities, Post-Digital Theory
\end{abstract}

\section{Introduction}

Contemporary art faces a persistent conceptual dilemma. Crisis has become a constant, undermining the subjective core of the concept, as it no longer records moments of decision-making during a transformation\footnote{Osborne, P. (2024). \textit{Crisis as Form: Contemporary Art's Temporal Politics}. London: Verso.}. Under these circumstances, contemporary artists exhibit a pronounced algorithmic nature in their extraction and reorganization of cultural resources: taking established cultural symbols and theoretical frameworks as input, they reassemble them using individualized processing rules, ultimately producing works that are superficially different but essentially similar. Contemporary art now must oversimplify in form, content, and intent, prioritizing ease of understanding and providing entertainment and eye-catching spectacle\footnote{Tornabuoni Art. (2024). "Rhythm and Repetition: A Contemporary Aesthetic." Exhibition documentation.}. This constitutes the core manifestation of the "conceptual collage syndrome."

The theoretical roots of this dilemma can be traced back to Walter Benjamin's insights into the fate of artworks in the age of mechanical reproduction. In the contemporary context of widespread artificial intelligence technology, this reproductive characteristic becomes even more pronounced, completely blurring the boundary between original and reproduction in AI-generated works\footnote{Benjamin, W. (1935). "The Work of Art in the Age of Mechanical Reproduction." In \textit{Illuminations}, trans. Harry Zohn. New York: Schocken Books.}.

It is against this background that the "Artism" project emerges, employing a dual-system AI architecture to explore the operational mechanisms of contemporary art's conceptual dilemmas while seeking new possibilities for artistic practice in the post-digital era.

\section{Literature Review}

\subsection{Contemporary Art's Conceptual Crisis and Creative Homogenization}

Contemporary art is trapped in an unprecedented conceptual crisis. As philosopher Peter Osborne points out, contemporary art faces a state of permanent crisis that no longer facilitates genuine transformation but instead destroys the subjective core of the crisis concept\footnote{Osborne, P. (2024). \textit{Crisis as Form}. London: Verso.}. In this state, most artists choose to operate within the existing contemporary art system through superficial cyclical operations, seeking differentiated expression through mechanical rearrangement and recombination of existing cultural symbols and art forms.

This phenomenon reflects what this research identifies as the \textbf{algorithmic characteristic} of contemporary artistic production: artists use established cultural symbols and theoretical frameworks as input, recombine them through individualized processing rules, and ultimately output works that appear differentiated on the surface but are essentially highly similar. This creative mode constitutes the core manifestation of the \textbf{conceptual collage syndrome}.

\subsection{AI Art and the Reconstruction of Aura}

The development of artificial intelligence technology has made the reproducible nature of artistic production even more pronounced. The emergence of AI-generated art calls for a reexamination of Walter Benjamin's original concept of "aura" and challenges existing notions of artistic authenticity and originality\footnote{Park, S. (2024). "The work of art in the age of generative AI: aura, liberation, and democratization." \textit{AI \& Society}, 39(2), 157-173.}. 

When art is co-produced with AI, the work is perceived as more innovative but less authentic and labor-intensive. Authenticity is a primary reason why audiences often view co-produced art less favorably\footnote{Gaskins, N. (2023). "The Aura of AI-Generated Art." \textit{Medium}, January 4, 2023.}. The ability of deep learning models to produce novel images across diverse subject matter and a wide range of artistic styles is beginning to expose the inadequacies of the ethical, aesthetic, epistemological, and legal frameworks we use to categorize art\footnote{AI-ARTS. (2024). "AI ART Competition 2024 - 3rd Edition." \url{https://ai-arts.org/ai-art-competition-2024-3rd-edition/}}.

\subsection{Post-Digital Context and Technological Infrastructure}

Florian Cramer's concept of "post-digital" describes an approach to digital media that no longer seeks technological innovation, but instead treats digitalization as a fait accompli and seeks to reconfigure it\footnote{Cramer, F. (2015). "What Is 'Post-Digital'?" In D. M. Berry \& M. Dieter (Eds.), \textit{Postdigital Aesthetics: Art, Computation and Design} (pp. 12-26). Palgrave Macmillan.}. In this state, digital technology ceases to exist as a novelty and becomes the infrastructure of everyday life.

Contemporary artistic practices are increasingly reflecting a focus on technological failures, glitches, and errors, which are incorporated into contemporary productions and offer new perspectives on art's conceptual dilemmas\footnote{Cascone, K. (2000). "The Aesthetics of Failure: 'Post-Digital' Tendencies in Contemporary Computer Music." \textit{Computer Music Journal}, 24(4), 12-18.}. This post-digital condition provides the technological and cultural context for understanding contemporary AI art practices.

\section{Project Content}

\subsection{AIDA: Artificial Intelligence Artist Database}

The idea for the AIDA (Artificial Intelligence Artist Database) project originated in 2019 and was profoundly influenced by the Italian artist collective "Anonymous" in 2022. Rather than a traditional database, AIDA is more like an AI arena disguised as social media—through multi-agent social simulations between historical artist NPCs, contemporary artist NPCs, virtual artist NPCs, and real users, it constructs a self-driven parallel world of art history and theory.

The system operates on a multi-agent architecture, with each agent type possessing unique training data and behavioral patterns:

\begin{itemize}
\item \textbf{Historical artist NPCs} undergo machine learning reconstruction based on archival materials, personal writings, and stylistic analysis, preserving individual characteristics while enabling generation of new works aligned with their aesthetic frameworks.

\item \textbf{Contemporary artist NPCs} undergo deep learning training based on the creative output and theoretical positions of active artists, providing dynamic representation of current artistic practice.

\item \textbf{Virtual prototype artists} act as novel aesthetic entities generated purely by AI, creating visual works using generative models like Flux as experimental variables within historical evolution scenarios.

\item \textbf{Human participants} interact through natural language processing interfaces, directly influencing the evolution of the simulated world.
\end{itemize}

AIDA's "madness effect" embodies the practice of speculative realism: as Graham Harman's object-oriented ontology argues, all objects possess mutual autonomy\footnote{Harman, G. (2018). \textit{Object-Oriented Ontology: A New Theory of Everything}. Pelican Books.}. Allowing Picasso to converse with ancient painters, allowing Van Gogh to create in the digital age—these "impossible" encounters are made possible through multi-model collaborative computing, breaking down established networks of relationships and creating new realities.

\subsection{IsmismMachine: Computational Critical Analysis System}

The IsmismMachine employs a systematic computational approach to critical analysis of contemporary art. Based on natural language processing and machine learning techniques, the system carries a clear critical thrust: investigating why contemporary art has become a form of "life-saving, remedial, and sham" through the "arbitrary" conceptual collages of artists establishing false "relationships" within Western art theory frameworks.

The system's core mechanism operates through reverse deconstruction of contemporary art theory texts. By semantically analyzing catalog structures of representative theoretical texts, the system extracts conceptual frameworks and discursive patterns, then algorithmically disassembles and reconstructs these concepts. This process reveals the "conceptual collage syndrome" in contemporary art theory production.

Key computational features include:

\begin{itemize}
\item \textbf{Conceptual entropy calculation}: measuring randomness and predictability of conceptual combinations for each work
\item \textbf{Critical label generation}: through automated text generation technology, systematically diagnosing theoretical appropriation and superficial innovation
\item \textbf{Algorithmic pattern analysis}: using neural networks to reveal the algorithmic production logic underlying seemingly original theoretical constructions
\end{itemize}

This phenomenon represents an extreme manifestation of the loss of aura in the age of mechanical reproduction—when artistic production shifts toward political practice, the uniqueness of creation is completely eroded\footnote{Benjamin, W. (1935). "The Work of Art in the Age of Mechanical Reproduction."}.

\subsection{Technological Dialectic of Dual Systems}

AIDA and the IsmismMachine embody different AI evolutionary mechanisms: AIDA employs a free-growth model based on reinforcement learning, generating emergent art historical narratives through random interactions; the IsmismMachine operates as a structured analytical system based on supervised learning, deconstructing artistic production logic through computational methods.

This dual-system architecture addresses the contemporary dilemma described by Byung-Chul Han: the shift from a disciplinary society to an "achievement society," where excessive affirmation replaces negativity\footnote{Han, B.-C. (2015). \textit{The Burnout Society}, trans. Erik Butler. Stanford University Press.}. In the art world, this manifests as an obsessive-compulsive drive for innovation—artists desperately seek "breakthroughs," but this anxiety actually limits their experimental spirit.

A dynamic data flow and feedback mechanism forms between the two systems: AIDA's virtual art historical evolution provides real-time analytical material for the IsmismMachine, while the latter's critical diagnosis influences the former's system parameters and agent behavior patterns, forming a self-reflective computational-critical cycle.

By employing AI agents as critical partners rather than mere production tools, the project explores new creative relationships in human-machine collaboration, offering a technological path for contemporary art to transcend conceptual dilemmas and restore the computational contemplation abandoned by achievement-driven society.

\section{Conclusion}

This literature review establishes theoretical foundations for understanding the intersection of artificial intelligence and contemporary art practice. Through the conceptual framework of "conceptual collage syndrome" and the practical implementations of AIDA and IsmismMachine, we identify both the critical challenges facing contemporary art and potential pathways forward.

The dual-system approach demonstrates how AI can serve not merely as a tool for artistic production, but as a medium for critical reflection and theoretical innovation. The project's deeper significance lies in offering possibilities for "slow thinking" in contemporary art, revitalizing artistic experimentation and critical reflection through technological means while addressing the fundamental question of how creativity and authenticity can be redefined in the age of artificial intelligence.

Future research will focus on the empirical analysis of system outputs and their impact on both artistic practice and theoretical discourse, contributing to broader discussions about the role of technology in creative human endeavors.

\end{document} 